% Equations
\newcommand{\beq}{\begin{equation}}
\newcommand{\eeq}{\end{equation}}
\newcommand{\bpsi}{\bar{\psi}}
\newcommand{\dslash}{\slashed{\partial}}
\newcommand{\Dslash}{\slashed{D}}
\newcommand{\Lagr}{\mathcal{L}}
\newcommand{\cpp}{\texttt{C++ }} 
\newcommand{\mpi}{\texttt{MPI }} 
\newcommand{\CIT}{{\color{red}(CITATION NEEDED)}}
\newcommand{\LINK}{{\color{red}(REFLINK NEEDED)}}
\newcommand{\NOTE}[1]{{\color{red} #1 }}
\newcommand{\FIGURE}[1]{{\color{red} FIG: #1 }}
\newcommand{\capt}[1]{\caption{\footnotesize{ #1 }}}
\newcommand{\D}{\mathcal{D}}
\newcommand{\Tr}{\mathrm{Tr}}
\newcommand{\fig}[4][1.0]{
    \begin{figure}[htp!]
        \begin{center}
            \includegraphics[scale=#1]{#2}
        \end{center}
        \capt{#3}
        \label{#4}
    \end{figure}
}
\newcommand{\bea}{\begin{align*}}
\newcommand{\eea}{\end{align*}}

\tikzset{->-/.style={decoration={
  markings,
  mark=at position #1 with {\arrow{>}}},postaction={decorate}}}
  \tikzset{-<-/.style={decoration={
  markings,
  mark=at position #1 with {\arrow{<}}},postaction={decorate}}}

  \tikzset{
    ncbar angle/.initial=90,
    ncbar/.style={
        to path=(\tikztostart)
        -- ($(\tikztostart)!#1!\pgfkeysvalueof{/tikz/ncbar angle}:(\tikztotarget)$)
        -- ($(\tikztotarget)!($(\tikztostart)!#1!\pgfkeysvalueof{/tikz/ncbar angle}:(\tikztotarget)$)!\pgfkeysvalueof{/tikz/ncbar angle}:(\tikztostart)$)
        -- (\tikztotarget)
    },
    ncbar/.default=0.5cm,
    }

\tikzset{square left brace/.style={ncbar=0.5cm}}
\tikzset{square right brace/.style={ncbar=-0.5cm}}

\tikzset{round left paren/.style={ncbar=0.5cm,out=120,in=-120}}
\tikzset{round right paren/.style={ncbar=0.5cm,out=60,in=-60}}


\newcommand{\tikzcuboidd}[4]{% width, height, depth, scale
\begin{tikzpicture}[scale=#4]
\foreach \x in {0,...,#1}
{   \draw (\x ,0  ,#3 ) -- (\x ,#2 ,#3 );
    \draw (\x ,#2 ,#3 ) -- (\x ,#2 ,0  );
}
\foreach \x in {0,...,#2}
{   \draw (#1 ,\x ,#3 ) -- (#1 ,\x ,0  );
    \draw (0  ,\x ,#3 ) -- (#1 ,\x ,#3 );
}
\foreach \x in {0,...,#3}
{   \draw (#1 ,0  ,\x ) -- (#1 ,#2 ,\x );
    \draw (0  ,#2 ,\x ) -- (#1 ,#2 ,\x );
}
\end{tikzpicture}
}

\newcommand{\tikzcube}[2]{% length, scale
\tikzcuboidd{#1}{#1}{#1}{#2}
}