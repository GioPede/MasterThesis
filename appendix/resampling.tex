The determination of the uncertainty of the results of a numerical experiment is Extremely important. In cases of small datasets, as in this work or in Lattice QCD in general, resampling methods are a popular solution for improving the estimate of the uncertainty. Another very important statistical property that is to be considered is the autocorrelation, introduced by the use of Markov chains.

\section{Resampling: Bootstrap and Jackknife}
Given a set of $N$ measurements of an observable $O$, a first estimate of the expectation value and of the uncertainty are the mean and the standard deviation of the data:
\beq
    \bar O = \frac{1}{N}\sum_{i=1}^N O_i, ~~~~~~ \sigma_O = \sqrt{\frac{1}{N}\sum_{i=1}^N (O_i - \bar O)^2}
\eeq
where $O_i$ are the measured values of the observable. Resampling methods use the data $O_i$ to construct multiple data series, and the average statistical properties of these series are used to estimate the properties of the original set. Both the bootstrap and the jackknife methods assume uncorrelated data. The observable $O$ can also be a derived quantity, such as a fit on some raw data.

\subsection{Bootstrap}
\label{app:resampling}
The Bootstrap method consists in building $N_b$ bootstrap samples by choosing $N$ points at random from the the original values (repetition are allowed). The average of a bootstrap sample is then:
\beq
    \hat O_b = \frac{1}{N}\sum_{i=1}^N O_{r(N)}
\eeq
where $r(N)$ stands for a random number between $1$ and $N$. One then has $N_b$ such averages and constructs:
\beq
    \tilde O = \frac{1}{N_b}\sum_{b=1}^{N_b} \hat O_b, ~~~~~~ \tilde\sigma_{\tilde O} = \sqrt{\frac{1}{N_b}\sum_{b=1}^{N_b} (O_b - \tilde O)^2}
\eeq
The expectation value of the observable is then $\langle O \rangle = \tilde O$ and its associated uncertainty is $\tilde\sigma_{\tilde O}$.

\subsection{Jackknife}
The Jackknife sample are not constructed at random, as is the case for bootstrap, but in a systematic manner. One creates $N$ jackknife samples by removing for each of them the $j$th element of the data:
\beq
\hat O_j = \frac{1}{N-1}\sum_{i\neq j}^N O_i
\eeq
One then computes:
\beq
\tilde O = \frac{1}{N}\sum_{j=1}^{N} \hat O_j, ~~~~~~ \tilde\sigma_{\tilde O} = \sqrt{\frac{N-1}{N}\sum_{j=1}^{N} (\hat O_j - \tilde O)^2}
\eeq
The final estimator for the expectation value is  $\langle O \rangle = \tilde O$ and the uncertainty is $\tilde\sigma_{\tilde O}$.\\

Both of these resampling methods have been implemented and used; they were found to be consistent with each other. The jackknife is deterministic, the bootstrap, for sufficiently large $N_b$ (usually $\approx 500$ are sufficient) had error estimates compatible with jackknife and of the same order of magnitude. The error estimate when using resampling compared to the standard deviation is smaller by orders of magnitude.

\section{Autocorrelation}
\label{app:autocorr}
The data generated via a Markov chain Monte Carlo method is always affected by autocorrelation because the data points intrinsically depend on each other. One can tune the algorithm to reduce the autocorrelation time until it becomes negligible, but this is not the case for Lattice calculations because of the computational cost involved.\\
We followed the procedures in \cite{dewitt-morette_monte_1997} to get an algorithm to estimate the integrated autocorrelation time and the error correction factor due to autocorrelation. \\
The autocorrelation function of an ordered set variables $\{x_1, x_2,\dots x_N\}$ with expectation value $\bar x$ is defined as:
\beq
    \Gamma(t) = \Gamma(-t) = \langle (x_i - \bar x)(x_{i+t} - \bar x)\rangle \approx \frac{1}{N-t}\sum_{i=1}^{N-t}  (x_i - \bar x)(x_{i+t} - \bar x)
\eeq 
where $t$ is the ``lag'' between two points. The integrated autocorrelation time is given by:
\beq
    \tau_{int} = \frac{1}{2} \sum_{t=1}^\infty \frac{\Gamma(t)}{\Gamma(0)} = \frac{1}{2} \sum_{t=1}^\infty \rho(t)
    \label{autocorr:inf}
\eeq
For sufficiently large $N$, the true variance of the data can be expressed as:
\beq
    \tilde\sigma^2 = 2\tau_{int}\sigma^2 
    \label{eq:tauint}
\eeq
This result is very neat, because it implies that in one can calculate the integrtated autocorrelation time correctly, the inclusion of the effects on the uncertainty estimate is trivial. In order to truncate the infinite summation in \cref{autocorr:inf} one can look at the deviation squared of $\rho(t)$, which as suggested by Sokal \CIT can be written as:
\beq
    \langle \delta \rho(t)^2\rangle \approx \frac{1}{N} \sum_{k=1}^\infty \left[ \rho(k+t) + \rho(k-t) - 2\rho(k)\rho(t)\right]^2
\eeq
all these terms, for a sufficiently large value of $k$ should all vanish, hence one can choose a cutoff $\Lambda$ and truncate the sum up to $t+\Lambda$. The integrated autocorrelation time, if the deviations of $\rho(t)$ become small, plateaus, so one can truncate the sum in \cref{autocorr:inf} as well: 
\beq
    \tau_{int} = \frac{1}{2} \sum_{t=1}^W \rho(t)
    \label{autocorr_time}
\eeq
where $W$ is the first lag $t$ for which $\rho(t) < \sqrt{ (\langle \delta \rho(t)^2\rangle}$, when the contribution to the integration of $tau_{int}$ from that lag become smaller than the deviation of that same lag. \\
The integrated autocorrelation time has been computed for all quantities presented in this work and statistical uncertainties associated with results have all been corrected via \cref{eq:tauint}