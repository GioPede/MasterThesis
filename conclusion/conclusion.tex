This work had two main goals: develop a completely new framework to perform lattice calculations in Yang-Mills theory, in particular using the gradient flow method; and to perform a calculation of the momentum scale $\Lambda$ by matching the perturbative analysis of the energy density as a function of flow time and the continuum limit extrapolation of lattice results.\\

The main features of Quantum Chromodynamics and of how to discretize it on a lattice have been discussed. In \cref{chap:code_design} the details for the implementation of the Metropolis algorithm and the numerical integration of the flow equation have been presented, together with the structure of the source-code that has been developed. \\
For the computational part of the work, the benchmarks and the performing scaling properties the program for both the generation of gauge field configurations and the gradient flow are a strong indication of the goodness of the implementation. \NOTE{MORE ON SCALING...}\\
The calibration of the parameters of the Metropolis algorithm for generating gauge field configuration was performed by looking at the integrated autocorrelation time $\tau_{int}$ of the topological charge at different flow times. It was found that, as expected, the autocorrelation  for smaller lattice spacings grows non-linearly. Consequently the natural solution has been to increase the distance in Monte Carlo time of the measurements of the observables, bu this meant a much larger computational cost so the ensemble size was also reduced. The final results for $\tau_{int}$ for the worse case ($\beta = 6.45$ and $a=0.0478$ fm) are contained. The energy density showed no particular issues in this regard.\\
When looking at the topological charge, computed on all of the four ensembles that were generated, it was found to be always consistent with 0, but given the very large volumes and the relatively small ensembles that were considered the uncertainties are rather large. The analysis of the topological susceptibility, on the other hand, proved to be a good estimate, the value we found is $\chi^{\frac{1}{4}} = 186.9(4.9)$ MeV, in good agreement with what reported in \cite{luscher_properties_2010,shindler_nucleon_2015} and also, through the Witten-Veneziano formula with the experimental result from the mass of the $\eta'$ meson.\\

Lastly, the main result of this work is the estimate of the momentum scale of Yang-Mills theory, starting from the lattice results for $t_f^2\langle E\rangle$, taking the continuum limit function and iteratively fitting the perturbative expansion function. The numerical procedure that has been applied produced results, depending on the order of loop correction for the renormalized coupling, that are consistent with other methods found in literature. The fact that the total uncertainty of our results is lower for the high order definitions of $\alpha_s(q)$ could mean that this method is valid and that is more precise, and one could add much simpler, than others. However it could also be the result of the underestimation of the systematic error, so further work is needed. 

\section{Future Work}
Most of the possible future work is related to the code-base for lattice calculations. Planned short-term improvements include: higher order calculations of gluonic observable (that is larger Wilson loops), improved actions such as the Conjugate Wilson Action or the Symanzik or Ywasaki improved actions, different sampling algorithms, like the Hybrid Monte Carlo algorithm. Another possible, but probably more time-consuming, new development is the introduction of dynamical fermions, allowing for full QCD calculations.\\
On a lower level, improvements in the handling of parallelization are planned. In particular some initial steps have been done in the creation of a low level library for SIMD calculations, using processor specific instruction sets. Hybrid memory parallelization, especially using the new features offered by the MPI 3.0 standard, is being considered as a possible feature.\\
    
Regarding the estimation of the momentum scale from lattice calculations, we are planning to test in the very near future the method that has been used in this work to data of the energy density of field configurations with $2+1$ dynamical fermions computed with the gradient flow method. The goal is to verify if the approach holds and if a value for $\Lambda_{QCD}$ can be obtained. \\
The most obvious future addition would be to perform the calculation on other ensembles of gauge field configurations, to check the stability with respect to lattice spacing, finite volume effects and, more in general, to improve the continuum limit extrapolation of the data. A more detailed analysis of the systematic error estimation should also be performed, to finally confirm the results presented in this thesis.