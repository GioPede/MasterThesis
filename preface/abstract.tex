The behavior of Quantum Chromodynamics, the quantum field theory that describes the strong interaction and the dynamics of quarks and gluons, at energies lower than a scale $\Lambda_{QCD} \approx 200-300$ MeV is not fully understood. The complexity of the problem stems from the exponential increase of the renormalized coupling constant towards low energies, which prevents a perturbative approach. Lattice QCD is a numerical method that has proven to be largely successful in describing the dynamics of strongly interacting quantum fields set on a discretized euclidean space time grid.\\

We developed a framework to perform lattice calculations of an $\mathrm{SU}(3)$ Yang-Mills theory with no dynamical fermion flavors. This theory, even if unphysical, presents most of the features of QCD and is generally a good approximation of it. In order to study divergent quantities and lattice discretization effects at low energies the recently proposed Gradient Flow method \cite{luscher_properties_2010} has been used and smoothing properties of the flow on gauge fields at non zero flow time are presented. \\
The procedure for the discretization of the field theory on a four-dimensional lattice is discussed as well as the main algorithms needed for the calculations. In particular, the Metropolis-Hastings algorithm, which is a Markov Chain Monte Carlo method, that can be used to generate statistical ensembles of lattice gauge field configurations will be introduced. Some of the more technical details of the structure of the program that has been developed are also shown.\\

The implementation of the Gradient Flow method is proven to give results for simple topological properties that are consistent with those found in the literature. These results, together with the analysis of the integrated autocorrelation time of the output data and the performance scaling properties of the program, suggest that the implementation is valid.\\
A method to estimate for the scale $\Lambda_{YM}$ of fermionless Yang-Mills theory starting from the lattice results for the dimensionless quantity $t_f^2\langle E \rangle$, where $t_f$ is the flow time and $\langle E \rangle$ the energy density, has been developed. The challenge of finding an unbiased region in the energy spectrum in which the continuum limit extrapolation of $t_f^2\langle E \rangle$ as a function of the energy scale $q=1/\sqrt{8t_f}$ matches the perturbative expansion of the same quantity was solved by systematically fitting to all energy ranges within a window. The final result, taken as the weighted median of the fit parameters using the fit quality as weights, was found to be in agreement, for all loop-order corrections considered, with the one found in the literature \cite{capitani_non-perturbative_1999} using an entirely different method.
