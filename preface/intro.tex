The theory that best describes the subatomic world as we know it is the Standard Model (SM) of particle physics. The interaction of quantum fields, the building blocks of the universe, are described within an extremely elegant mathematical framework. The particles, as they are most commonly intended, are nothing but ripples, local excitations, of the fields.\\
The strong interaction is one of the components of the Standard Model and arguably it is the most complicated of all. Quantum Chromodynamics (QCD) is the quantum field theory that describes it; the particle fields it considers are the quarks and the gauge fields (the mediators of the force) are the gluons. The reason why it is so difficult to study is because the ``strength" of the force, given by the coupling constant, changes rapidly at different energies. The high-energy limit is relatively simple to describe: perturbative expansions in powers of the coupling constant can be performed, leading to advanced analytical results, for example the calculation of cross sections for collision processes. These computations are the key ingredient for understanding the phenomenology of particle accelerators.\\
Studying the low-energy limit of the strong interaction is a completely different problem. In a larger picture, the whole branch of physics called nuclear structure is fundamentally investigating the nature of the residual strong interactions between nucleons, protons and neutrons. These studies can be approached experimentally, by using data collected at Rare Isotope Beams facilities, to construct effective potentials between nucleons. These results are then used in many-body quantum mechanical simulations to get better and better insight on the underlying theory. Other approaches start from Effective Field Theories to model from ``first principles'' the nuclear interactions.\\
One of the biggest challenges in the theory of strong interactions is to link the description given by nuclear theory to the underlying Quantum Chromodynamics fundamental theory. The main problem is that the coupling constant of QCD becomes large at low energies. This makes perturbative calculations impossible to be carried out. Numerical methods are one of the possible solutions to the problem.\\

Lattice QCD is the most successful description of the strong forces at low energies that we know today. The key idea is to discretize spacetime in a four-dimensional grid of points. Quarks are set on sites of this lattice and physical properties are extracted by considering discretized paths between different lattice sites. From the time it was first developed by K.G.Wilson, Lattice QCD has evolved dramatically, largely due to the increase of computational power over the years. Some of the most interesting calculations are still not technically feasible with the computational resources of today, this makes Lattice QCD a prominent field in the context of High Performance Computing (HPC). \\

\section{Goals and Motivation}
As this work is intended to be a Computational Physics thesis, the final goals of the project are dual. As far as computational science is concerned, the aim was to develop a computer program that implemented the rather recently proposed Gradient Flow Method \cite{luscher_properties_2010} for Lattice QCD. This has been done with the idea of laying the foundations for a possible future more complete framework for performing lattice calculations. This larger program, however, is clearly too much to develop in the time frame of a master thesis, so the first building blocks have been implemented in this work.\\
The resulting lattice simulation program has been used to analyze a physical problem related to the scale $\Lambda_{YM}$ of Pure Yang-Mills Theory. By that we mean the quantum field theory based on the $\mathrm{SU}(3)$ gauge symmetry group, the same of QCD, but with no dynamical fermion flavors. It is an interesting first approximation of the full theory, that however requires a simpler numerical implementation. \\
The specific problem we investigated regarded the matching between lattice calculations and analytical results from perturbation theory in the context of the Gradient Flow Method. We propose an estimate of the scale of Pure Yang-Mills theory starting from the lattice computation of the dimensionless quantity $t_f^2\langle E\rangle$, which is related to the action of the gauge fields as they are evolved by the gradient flow equation. The same object has a known perturbative expansion at an energy scale dependent from the fictitious dimension $t_f$, the flow time, that we try to fit to the data extracted from our numerical simulation to estimate $\Lambda_{YM}$. \\
Other methods to compute the scale of Pure Yang-Mills theory have been previously studied in literature. The method we propose could be a simpler alternative to those. Future work is planned to extend the results of this thesis to the case of QCD and its scale $\Lambda_{QCD}$, using gauge field configurations generated with more sophisticated Lattice QCD codes, to check the validity of the method.

\section{Our Contribution}
This thesis, as every other scientific work, has its foundation in many other previously published work. It is then worth to state clearly what the author has contributed to. \\
The main contribution is the development of a completely novel Lattice code: starting from the very basics of $\mathrm{SU}(3)$ matrix arithmetics; to the sampling of the space of gauge field configuration via the Metropolis algorithm and finally to the implementation of a numerical integrator for the gradient flow equation. \\
There exist several other professional Lattice QCD code bases that are well-tested and highly optimized. The aim of this work is not to compete with these packages, but rather to gain some insight into the key methods and strategies as well to understand the challenges that the problem poses. The development process resulted in approximately 10000 lines of code written in \cpp and this required most of the time spent on the thesis.\\
The second contribution can be summarized in the analysis of the feasibility of the estimation of $\Lambda_{YM}$ from data collected with the lattice implementation of the gradient flow. This new method, which appears to produce consistent results, could be the subject of future studies. 


\section{Structure of the Thesis}
This work is structured into three parts. In \cref{part:intro} we will discuss the general features of QCD, \cref{chap:qcd_intro}, to give the reader some of the basics of this complicated and very diverse theory. Elements of the discretization process needed to develop Lattice QCD and some basic definitions will be provided in \cref{chap:lattice_intro}, while in \cref{chap:grad_intro} more advanced topic such as the gradient flow and the reference scale fixing in Lattice QCD will be discussed.  \\
In \cref{part:implementation} we introduce the numerical implementation we developed by presenting the main algorithms that have been used and the general structure of the code, \cref{chap:code_design}. In the following chapter the tests that have been performed to set up the algorithms and to check the performance scaling are presented.\\
Finally, in \cref{part:results} we first give an overview of the observables that have been computed on the ensembles of gauge field configurations that we have generated, \cref{chap:obs_results}. Then  in \cref{chap:advance_results} we present the procedure that has been used to estimate the scale $\Lambda_{YM}$ and the final results that have been obtained. 