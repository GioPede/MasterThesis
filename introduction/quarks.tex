\begin{figure}[!htb]
\centering
\pgfmathsetmacro{\dist}{2}
\pgfmathsetmacro{\disttwo}{3}
\pgfmathsetmacro{\distthree}{1.5}
\pgfmathsetmacro{\outerR}{0.6}
\pgfmathsetmacro{\v}{30}
\pgfmathsetmacro{\sep}{4}
\definecolor{mygreen}{rgb}{0, 0.5, 0}
\begin{tikzpicture}[>=latex,x=0.8cm,y=0.8cm]
    \filldraw[fill=red] (0,0) circle (0.5);
    \filldraw[fill=mygreen] (0,0) ++(135:\dist) circle (0.5);
    \draw (0,0) ++(45:\outerR) -- ++(135:\dist) arc(45:225:\outerR) -- ++(315:\dist) arc(-135:45:\outerR);

    \begin{scope}[shift={(\sep,0)}]
        \filldraw[fill=red] (0,0) circle (0.5) coordinate (A);
        \filldraw[fill=mygreen] (0,0) ++(135:\disttwo) circle (0.5) coordinate(B);
        \draw (A) ++(45-180-\v:\outerR) coordinate(x1) arc(45-180-\v:45+\v:\outerR) coordinate (x2);
        \draw (B) ++(45-\v:\outerR) coordinate(y2) arc(45-\v:45+180+\v:\outerR) coordinate (y1);
        \path (x2) edge[bend left=\v] (y2);
        \path (x1) edge[bend right=\v] (y1);
        \draw (0,0) ++(135:\distthree) coordinate (X);
        \draw (-\sep/2,0) node (Y) {Energy};
        \path[thick][->] (Y.north) edge[bend left] (X);
    \end{scope}

    \begin{scope}[shift={(2*\sep,0)}]
        \filldraw[fill=red] (0,0) circle (0.5) coordinate (A);
        \filldraw[fill=mygreen] (0,0) ++(135:\distthree) circle (0.5) coordinate(B);
        \draw (A) ++(45-180-\v:\outerR) coordinate(x1) arc(45-180-\v:45+\v:\outerR) coordinate (x2);
        \draw (B) ++(45-\v:\outerR) coordinate(y2) arc(45-\v:45+180+\v:\outerR) coordinate (y1);
        \path (x2) edge[bend left=\v] (y2);
        \path (x1) edge[bend right=\v] (y1);
    \end{scope}
    \begin{scope}[shift={({2*\sep-(\distthree+0.25)*cos(45)},{(\distthree+0.25)*sin(45)})}]
        \filldraw[fill=red] (0,0) circle (0.5) coordinate (A);
        \filldraw[fill=mygreen] (0,0) ++(135:\distthree) circle (0.5) coordinate(B);
        \draw (A) ++(45-180-\v:\outerR) coordinate(x1) arc(45-180-\v:45+\v:\outerR) coordinate (x2);
        \draw (B) ++(45-\v:\outerR) coordinate(y2) arc(45-\v:45+180+\v:\outerR) coordinate (y1);
        \path (x2) edge[bend left=\v] (y2);
        \path (x1) edge[bend right=\v] (y1);
    \end{scope}
    \begin{scope}[shift={(2*\sep,0)}]
        \draw (0,0) ++(135:\distthree) coordinate (X);
        \draw (-\sep/2,0) node (Y) {Energy};
        \path[thick][->] (Y.north) edge[bend left] (X);
    \end{scope}

    \begin{scope}[shift={(3*\sep,0)}]
        \filldraw[fill=red] (0,0) circle (0.5);
        \filldraw[fill=mygreen] (0,0) ++(135:\dist) circle (0.5);
        \draw (0,0) ++(45:\outerR) -- ++(135:\dist) arc(45:225:\outerR) -- ++(315:\dist) arc(-135:45:\outerR);
    \end{scope}
    \begin{scope}[shift={({3*\sep-(\dist+1.5)*cos(45)},{(\dist+1.5)*sin(45)})}]
        \filldraw[fill=red] (0,0) circle (0.5);
        \filldraw[fill=mygreen] (0,0) ++(135:\dist) circle (0.5);
        \draw (0,0) ++(45:\outerR) -- ++(135:\dist) arc(45:225:\outerR) -- ++(315:\dist) arc(-135:45:\outerR);
    \end{scope}
\end{tikzpicture}
\caption{Representation of confinement using flux-tubes . As two color sources are pulled apart, the energy stored in the color field between the sources increases so much that a $q\bar q$ pair is formed from the vacuum energy.}
\label{intro:flux-tubes}
\end{figure}