In the case of highly non-linear quantum field theories it can be useful to study the properties of flows in field space. The key idea is to see how the theory evolves as it becomes less local, by driving it to the stationary points of the action.\\
For the $SU(3)$ gauge field case, starting from a field $A(x)$ the characterizing equations are:
\beq
    \begin{aligned}
        &\dot{B}_\mu = D_\mu G_{\mu\nu}\\
        &G_{\mu\nu} = \partial_\mu B_\nu - \partial_\nu B_\mu + [B_\mu, B_\nu]  
    \end{aligned}
\eeq
where the $B(t_f, x)$ field is the flowed version of the original field at flow-time $t_f$. This is imposed by fixing:
\beq
    B_{\mu|t_f = 0} = A_\mu
\eeq
The covariant derivative is extended to represent the derivative of the field $B$ instead of $A$, this leaves the simple definition for the non-flowed field from the boundary condition. It is generalized straightforwardly as:
\beq
    D_{\mu} = \partial_\mu + [B_\mu, \cdot ]
\eeq
In a discretized theory taking the derivatives is not a trivial procedure, but we can use instead the intuitive idea of the flow equations, that is to use the steepest descent method on the action of a field. We then introduce the flowed lattice gauge field $V_{t_f}(x,\mu)$ as the flowed version of a field $U(x,\mu)$. Our flow equation then becomes: 
\beq
    \dot V_{t_f}(x,\mu) = - g_0^2 [\partial_{x,\mu}S_G(V_{t_f})]V_{t_f}(x,\mu)
\eeq  
where $S_G$ is the Wilson action, as defined in \ref{wilsonaction}, generalized to flowed fields. The generalization is really straightforward, as it only requires to compute the Wilson loops on the flowed gauge field.

\section{Perturbative Analysis of the Wilson Flow}
An important thing to consider when applying the Wilson flow to a field is the renormalization of the observables: one has to check that expectation values of observables at non-zero flow-time are renormalized quantities. Following the calculations performed in \CIT luscher, we will consider the energy as our base observable. \\
First, we note that the flow equation is invariant  under flow-time independent gauge transformations, this prevents a detailed study of the renormalization. So we consider a modified version of the flow equation by adding one term:
\beq   
\dot{B}_\mu = D_\mu G_{\mu\nu} + \lambda D_\mu\partial_\nu B_\nu
\eeq
the original case is obtained again by setting $\lambda =0$ and considering:
\beq   
    B_\mu = \Lambda B_\mu|_{\lambda=0} \Lambda^{-1} +  \Lambda \partial_\mu \Lambda^{-1}
\eeq
and with $\Lambda(t,x)$ now being a flow-time dependent gauge transformation, set by:
\beq   
\dot{\Lambda}_\mu = -\lambda\partial_\nu B_\nu\Lambda~~~\text{with}~~\Lambda|_{t_f=0} = 1
\eeq 
