Quantum Chromodynamics, commonly referred to as QCD, is the quantum field theory that describes the behaviour of strongly interacting matter, that is quarks and gluons. It is a non-abelian gauge theory based on a $SU(3)$ symmetry group. In this chapter we will discuss how the QCD lagrangian density is obtained, how it links with the Standard Model of particle physics ($SM$) and some of the major results of the thoery. \cite{peskin}

\section{Derivation of the QCD Largangian}
Starting from a complex fermion field $\psi(x)$ we can construct a vector of $N$ such fields. The case of the strong force is that of $N=3$:
\begin{equation}
  \psi(x) = \begin{pmatrix}
      \psi_a(x)\\
      \psi_b(x)\\
      \psi_c(x)
  \end{pmatrix}
\end{equation}
We then construct a lagrangian density starting from Dirac's equation:
\begin{equation}
  \Lagr = i\bpsi(\dslash - m )\psi
\end{equation}
a change
