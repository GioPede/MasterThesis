The Standard Model of particle physics (SM) is the theory of fundamental particles and their interactions. Three of the four known fundamental forces are described by it, the exception is gravity, so all phenomena regarding the Electromagnetic, Weak and Strong forces are included within the theory. The SM describes particles (fermions) which can be quarks or leptons, that interact through the action mediated by gauge fields (bosons) that carry the forces, the photon, gluons, W and Z are such bosons.\\ 
%\fig{introduction/SM.png}{Summary of the particles on the Standard Model}{intro:SM}
\begin{center}
\input{introduction/SM}  
\end{center}
From a group theory point of view, the SM is the composition of three different symmetry groups, each associated with a fundamental force:
\beq
    SU(3)_C \times \underbrace{SU(2)_L \times U(1)_Y}_{broken~to~~ SU(2)_W \times U(1)_{EM}}
\eeq
The $U(1)_{EM}$ symmetry group is associated with the electromagnetic interaction and derives from the spontaneously broken symmetry $U(1)_{Y}$, hyper-charge, that together with the $SU(2)_L$ through the Higgs mechanism recover also the term is associated with the weak force.\\ 
Quantum Chromodynamics, commonly referred to as QCD, is the quantum field theory, contained in the SM, that describes the behavior of strongly interacting matter, that is quarks and gluons. It is a non-abelian gauge theory based on a $SU(3)$ symmetry group. In this chapter we will discuss the QCD lagrangian density, its properties and some of the major results of the theory. Some intermediate knowledge of Quantum Field Theory is assumed and derivations and proofs mainly follow the reasoning found in \cite{peskin}.

\section{The QCD Lagrangian} 
In quantum field theory the characterizing equation of a theory is its lagrangian density, because it contains all the information about the fields that are involved, their properties and most importantly, their interactions. For QCD the simplest form is based on the Yang-Mills Lagrangian, with a $SU(3)$ symmetry group:  
\beq
  \Lagr_{QCD} = -\frac{1}{4}(G_{\mu\nu}^a)^2 + \sum_{f=1}^{n_f}\bpsi_f(i\Dslash - m_f)\psi_f
  \label{lagr:qcd}
\eeq 
Here $\psi_f$ represents the complex-valued fermion field of flavor $f$, with mass $m_f$. These fields are the quark fields that come in six flavors: $u$ (up), $d$ (down), $s$ (strange), $c$ (charm), $b$ (bottom) and $t$ (top).  The second element in the lagrangian is the Gluon Field Strength Tensor, $G_{\mu\nu}^a$. The two indices $\mu$ and $\nu$ are Lorentz indices and $a$ is the index of the generators of the gauge group, $SU(3)$ in this case. Note that Einstein summing convention on repeated indices is implicit, for example when taking the square of the field strength tensor three sums are applied. The definition of $G_{\mu\nu}^a$ is:
\beq
  G_{\mu\nu}^a = \partial_\mu A_\nu^a - \partial_\nu A_\mu^a + gf^{abc}A_\mu^b A_\nu^c 
\eeq
in this equation $A_\mu^a $ is the gluon field, that carries a Lorentz index and a group generator index and the $f^{abc}$ are the structure constants of $SU(3)$, which satisfy: 
\beq
  [t^a, t^b] = if^{abc}t^c
\eeq
with $t^a$ being the generators of the algebra $\mathfrak{su}(3)$. The covariant derivative $\Dslash$ is defined then as:
\beq
    \Dslash = \gamma^\mu \partial_\mu  - ig\gamma^\mu t_aA_\mu^a
\eeq
With the information contained in the lagrangian density the behavior and the interactions of all particles are set.


\subsection{Feynman Rules of QCD}
A key element that is needed to perform perturbative calculations in a quantum field theory are Feynman Rules. These are a set of equations and rules that represent the propagation of fields and the interaction vertices of the theory. For the case of QCD, being a non-abelian gauge theory, some vertices represent interactions between gauge bosons only, as opposed to Quantum Electrodynamics, QED, that forbids photon-photon interactions. \\
To begin with, we need to write out all of the terms of the lagrangian density individually. We assume only one quark flavor as no term in the lagrangian can change this quantum number.
\begin{align}\nonumber
    \Lagr_{QCD} &= -\frac{1}{4}(G_{\mu\nu}^a)^2 +\bpsi(i\Dslash - m)\psi \\\nonumber
    &= -\frac{1}{4}\left( \partial_\mu A_\nu^a - \partial_\nu A_\mu^a + gf^{abc}A_\mu^b A_\nu^c \right) \left(\partial_\mu A_\nu^a - \partial_\nu A_\mu^a + gf^{abc}A_\mu^b A_\nu^c \right) \\\nonumber
    &~~~+ \bpsi(i\gamma^\mu \partial_\mu  + g\gamma^\mu t_aA_\mu^a - m)\psi \\\nonumber
    &= -\frac{1}{4}\left(\partial_\mu A_\nu^a - \partial_\nu A_\mu^a\right)^2 \\\nonumber
    &~~~+ \frac{1}{2}f^{abc}\left(\partial_\nu A^a_\mu - \partial_\mu A^a_\nu \right)\left[ A^{b\mu}, A^{c\nu} \right] \\
    &~~~-\frac{1}{4}g^2f^{abc}f^{ade}A_\mu^bA_\nu^cA^{d\mu}A^{e\nu} \label{lagr:expanded}  \\\nonumber
    &~~~+g\gamma^\mu t_a A_\mu^a \bpsi\psi \\\nonumber
    &~~~+ \bpsi(i\gamma^\mu\partial_\mu - m)\psi
\end{align}

First we define the gluon and quark propagators, which can be obtained from the first and last terms respectively:
\begin{center}
  \vspace{-1cm}
  \begin{tabular}{cl}       
    \feynmandiagram [horizontal=a to b] {
        a [particle=\(a~\mu\)] -- [gluon, momentum=\(k\)]  b [particle=\(b~\nu\)] ,
        }; & 
        $ = -\frac{i \delta_{ab}g^{\mu\nu}}{k^2} $\\
    \feynmandiagram [horizontal=a to b] {
        a [particle=\(i\)] -- [fermion, momentum=\(k\)]  b [particle=\(j\)] ,
        }; & 
    $ = \frac{i \delta_{ij}}{\slashed{k}-m} $\\
  \end{tabular}
\end{center}
The third term in \ref{lagr:expanded} represents a three gluon vertex:
\begin{center}
\begin{minipage}{0.4\textwidth}
    \hspace{2cm}\feynmandiagram [horizontal=a to b] {
        b[particle=\(a~\mu\)]  -- [gluon, momentum=\(p\)]  a[dot],
        a1[particle=\(b~\nu\)]  -- [gluon, momentum=\(q\)]  a,
        a2[particle=\(c~\rho\)]  -- [gluon, momentum=\(k\)]  a,
        };
\end{minipage}
\begin{minipage}{0.58\textwidth}
    $ = gf^{abc}\left[g^{\mu\nu}(p-q)^\rho + g^{\nu\rho}(q-k)^\mu + g^{\rho\mu}(k-p)^\nu \right] $
\end{minipage}
\end{center}

Then the four-gluon vertex:
\begin{center}
    \begin{minipage}{0.4\textwidth}
        \hspace{2cm}\feynmandiagram [horizontal=a1 to b1]{
            a1[particle=\(b~\mu\)]  -- [gluon, momentum=\(p\)]  a[dot] -- [gluon]  b1[particle=\(c~\nu\)],  
            a2[particle=\(d~\rho\)]  -- [gluon, momentum=\(k\)]  a -- [gluon]  b2[particle=\(e~\sigma\)],
            b1 -- [gluon, momentum=\(q\)] a, 
            b2 -- [gluon, momentum=\(r\)] a, 
            };
    \end{minipage}
    \begin{minipage}{0.58\textwidth}
        \begin{align}\nonumber = -ig^2&
        [~~~f^{abe}f^{acd}(g^{\mu\nu}g^{\sigma\rho} - g^{\mu\rho}g^{\nu\sigma}) \\\nonumber
        &+f^{abd}f^{ace}(g^{\mu\nu}g^{\sigma\rho} - g^{\mu\sigma}g^{\nu\rho}) \\\nonumber 
        &+f^{abc}f^{aed}(g^{\mu\sigma}g^{\nu\rho} - g^{\mu\rho}g^{\nu\sigma})]
        \end{align}
    \end{minipage}
\end{center}
Finally we have the gluon-quark interaction vertex:
\begin{center}
    \begin{minipage}{0.4\textwidth}
        \hspace{2cm}\feynmandiagram [horizontal=a1 to b1]{
            a1[particle=\(a~\mu\)]  -- [gluon, momentum=\(p\)]  b1[dot],
            c[particle=\(i\)] -- [fermion, momentum=\(q\)] b1 -- [fermion, momentum'=\(k\)] d[particle=\(j\)]
            };
    \end{minipage}
    \begin{minipage}{0.58\textwidth}
        \hspace{-1.5cm}\begin{align}\nonumber = 2igt^a\gamma^\mu
        \end{align}
    \end{minipage}
\end{center}
Note that the lagrangian we considered, and the resulting Feynman rules, is over-simplified: the ``full'' lagrangian contains terms from Faddeev-Popov ghosts and counter-terms from the renormalization procedure. Nevertheless, interesting qualitative features can be inferred from the Feynman rules: the fact that gluons interact with each other through the three- and four-gluon vertices; the non flavor-changing interaction between gluons and quarks, which decouples completely different quark flavors; the ``color-changing nature'' of the quark-gluon interaction, given by the $t^a$ matrix in the vertex term that shows how the color state of a fermion is changed by the absorption or emission of a gluon.\\


\subsection{Gauge Symmetry of the Lagrangian}
\label{intro:symmetry}
The QCD lagrangian must be gauge invariant to be physical. The concept of gauge invariance is crucial in the construction of a discretized lattice theory from the continuum one. Let's consider a quark field $\psi(x)$ and a local gauge transformation in the internal space of $SU(3)$ applied to it:
\beq
    \psi(x) \rightarrow \psi'(x) = U(x)\psi(x)~~~:~~U(x)=\exp(i(\alpha^a(x)t^a))
\eeq
here $U(x)$ is an element of the $SU(3)$ gauge group. This transformation if applied to the simple Dirac free-field Lagrangian would generate an additional term:
\begin{align} \label{lagr:dirac}
\Lagr_{Dirac} =\bpsi(i\gamma^\mu\partial_\mu - m)\psi \rightarrow \Lagr'_{Dirac} &= \bpsi U^\dagger(i\gamma^\mu\partial_\mu - m)(U\psi) \\\nonumber &= \bpsi'(i\gamma^\mu\partial_\mu - m)\psi' + i\bpsi'\gamma^\mu\psi (\partial_\mu U)  
\end{align} 
In order to fix this problem a new field $A_\mu(x)$, the gauge field, is introduced and it enters the definition of the covariant derivative, so that $\partial_\mu$ becomes $D_\mu = \partial_\mu - igA_\mu(x)$ as previously stated. The transformation rule for $A_\mu(x)$ is fixed in order to cancel the extra term in \ref{lagr:dirac} exactly, such that:
\beq
    D_\mu\psi \rightarrow   (D_\mu\psi)'= (\partial_\mu - igA'_\mu(x))\psi' = U(D_\mu\psi)
\eeq
and this fixes the transformation for $A_\mu(x)$ to be:
\beq
    A_\mu(x) \rightarrow A'_\mu(x) = U\left[ A_\mu(x) - \frac{i}{g} U^\dagger\partial_\mu U \right] U^\dagger
\eeq
we can see that the last term contributes effectively to the lagrangian as $-i\bpsi'\gamma^\mu\psi (\partial_\mu U)$, which is what we want to cancel. For an infinitesimal transformation we can expand the matrix $U$ and get the following for the gauge field:
\beq
A^a_\mu(x) \rightarrow A'^a_\mu(x) = \alpha^a(x) - \frac{i}{g}\partial\alpha^a(x) + f^{abc}\alpha^b(x)\alpha^c(x)
\eeq
which is the last element needed. One can now look at all the possible gauge invariant objects that can be constructed with the fields $\psi$ and $A$ of order 4, the same of the lagrangian. Apart from the one already present in the Dirac lagrangian with the covariant derivative, there are only two additional terms that are gauge invariants and of order 4 and they both can be taken by considering the gauge field tensor:
\beq
G^a_{\mu\nu} \equiv \frac{i}{g} \left[D_\mu,D_\nu \right] =  \partial_\mu A_\nu^a - \partial_\nu A_\mu^a + gf^{abc}A_\mu^b A_\nu^c 
\eeq 
this is clearly gauge invariant since it is commutator of covariant derivatives and it has dimension 2. One can now construct the gauge field kinetic term, the well known $-\frac{1}{4}(G^a_{\mu\nu})^2$ and reconstruct \ref{lagr:qcd}. However, there is an additional term, not included in the QCD lagrangian, that is the "dual term", or "theta term", which will be interesting further in the work. It is defined as $\theta G^a_{\mu\nu}\tilde G^{a\mu\nu}$ where $\tilde G^{a\mu\nu} = \epsilon_{\mu\nu\rho\sigma}G^{a\rho\sigma}$ is the dual of the field tensor. With it the lagrangian becomes:
\beq
\Lagr_{QCD} = -\frac{1}{4}(G_{\mu\nu}^a)^2 + \bpsi(i\Dslash - m)\psi + \theta G^a_{\mu\nu}\tilde G^{a\mu\nu}
\eeq
The theta term is usually neglected because there is no experimental evidence of it, but in principle it cannot be excluded. It is the simplest $CP$ violating term that can be added to the QCD lagrangian and for this is of particular interest in the study of $\slashed{CP}$ phenomena, like the nucleon electric dipole moment (EDM). 

\section{General Properties of QCD}
Quantum Chromodynamics exhibits a set of features as a theory that are common to all non-abelian gauge theories. We have already seen one, that is the direct interaction of the gauge bosons, something that is not allowed in abelian theories as QED. Other interesting properties emerge when trying to renormalize the theory and are in general linked to the fixing of the scale, which leads to the concept of running coupling. In the particular case of QCD the coupling constant at in a low-energy regime leads to Confinement, while in the high-energy limit Asymptotic Freedom emerges. The discovery of asymptotic freedom by Gross, Wilczek  and Politzer \cite{Gross-Wilczek}\cite{Politzer},  was used as an indication that QCD is indeed the correct theory of the strong interaction.

\subsection{Running Coupling} 
The beta function of the coupling constant, $\beta(g)$ defines the rate at which the renormalized coupling varies as the renormalization scale $Q$ changes.
\beq
    \beta(g) = \frac{d}{d\log(\mu)}g(\mu) 
    \label{beta:log}
\eeq 
For a generic abelian theory $SU(N)$ one can expand the $\beta$ function in orders of $g$ as:
\beq
    \beta(g) = \beta_0g^3 + \beta_1g^5 + \beta_2g^7 + \dots   
\eeq
One can then integrate up to arbitrary order \ref{beta:log} and get an expression for $g(\mu)$. The coefficients $\beta_i$ are obtained from computing contribution of higher and higher diagrams to the coupling. The value of $\beta_0$, from 1-loop corrections, is:
\beq 
    \beta_0 = - \frac{g^3}{(4\pi)^2} \left( \frac{11}{3}N - \frac{2}{3}n_f \right)
    \label{beta:qcd}
\eeq
the minus sign in front implies that any abelian theory with a sufficiently small number of fermions, less than $\frac{33}{2}$ (that is 16 in practice),  is asymptotically free.\\
The renormalization group equation is usually expressed in terms of the analogous of the fine structure-constant for the strong force, $\alpha_s = g^2/4\pi$. The first order solution is given by plugging \ref{beta:qcd} into \ref{beta:log}. In terms of $\alpha_s$ at a scale $\mu$ we get:
\beq
    \alpha_s(\mu) = \frac{\alpha_s}{1 + \frac{\beta_0\alpha_s}{4\pi}\log(\mu^2/M^2)}
    \label{alpha:M}
\eeq  
The typical choice for $\mu$ is the mass of the $Z$ boson, where QCD can be compared relatively simply to the other forces. The coupling depends on an arbitrarily chosen renormalization point $M$. A convenient choice is to define a mass scale $\Lambda$ that satisfies:
\beq
    1 = g^2(\frac{\beta_0}{4\pi^2})\log(M^2/\Lambda^2) 
\eeq  
\NOTE{check constants...} This simplifies \ref{alpha:M} to its well-known form, correct up to one loop corrections:
\beq
    \alpha_s(Q^2) = \frac{4\pi}{\beta_0\log(\mu^2/\Lambda^2)}
    \label{alpha:1loop} 
\eeq
In a later section \LINK we will show a more precise result, correct up to order 4, of this result. Our main goal for this thesis is to find a simple and not expensive way to estimate the mass scale parameter $\Lambda$ from lattice calculations. In \ref{fig:running} a higher order approximation of $\alpha_s(\mu)$ is plotted against some experimental values. 
\fig{introduction/running-coupling.png}{The strong coupling as a function of the energy scale. Image from \cite{coupling-Chaudhuri} }{fig:running}
Now we will talk briefly of what this result implies.  

\subsubsection{Asymptotic Freedom}
Equation \ref{alpha:1loop} is a clear indication that QCD at high energies has a small coupling. Asymptotic freedom is the property of gauge theories, QCD is usually the example for it, that causes the interactions between the fields do become weaker as the energy scale increases. This is for example one of the basis on which unification theories are based on, the fact that there exists an energy scale at which the strong force has a coupling equivalent to the one of the electroweak interaction.  \\
Asymptotically free theories, can be analyzed perturbatively at sufficiently large energies and are believed to be consistent up to any energy scale.

\subsubsection{Confinement}
At low energy scales from \ref{alpha:1loop} we can infer that the coupling constant increases exponentially and approaches 1. This is the reason for the fundamentally different nature of the strong force compared to the other forces, no perturbative expansion can be made at low energies. The exact proof of how this links with the color confinement of QCD is yet not known, but qualitatively it can be explained by the fact that the gauge bosons of the theory, the gluons, carry color charge just like the quarks.
In general color charged particles cannot be isolated, so that quarks and gluons are not detectable alone, but always in the from of hadrons: colorless objects formed of multiple quarks, mesons (quark-anti-quark) and hadrons (three quarks or three anti-quarks). Also glueballs, combinations of gluons such that the total is colorless, are in principle allowed, but have not been observed yet. Confinement is the phenomenon for which it is not possible to isolate a color charge, single quarks or gluons, from a hadron without producing other new hadrons. Single colored particle in a very small time scale undergo hadronization, the process of spawning new quarks or anti-quarks from the vacuum to balance the total color charge and produce colorless matter. \\
The other usual picture is to consider the gluons being exchanged by two quarks to from flux-tubes that, if stretched by separating the quark sources, eventually store enough energy to make a quark-anti-quark pair energetically favorable, as depicted in figure \ref{intro:flux-tubes}
\fig[0.4]{introduction/flux-tubes.jpg}{Representation of confinement using flux-tubes. As two colour sources are pulled apart, the energy stored in the colorfield between the sources increases so much that a $q\bar q$ is formed at some point.}{intro:flux-tubes}

\section{Methods and Regimes of Chromodynamics}
Given the very different behaviors of the strong force at different energy scales, QCD needs to be dealt with in various ways depending on the scale of interest. At high energies perturbation theory can be applied safely, but at low energies no expansion in the coupling constant can be made. 

\subsubsection{Perturbative QCD}
At high energies, like the scale of the large particle accelerators that are colliders currently available, the QCD coupling constant is sufficiently small to allow analytical calculations of Feynman diagrams to be meaningful. Hadronization is neglected as the time scale is small enough to consider it a post-collision process. For example, at a scale where the strong force is comparable to the electroweak interaction, the cross section of $e^+e^-\rightarrow \text{hadrons}$ compared to that of $e^+e^-\rightarrow \mu^+\mu^-$ can be used to measure the number of quark flavors that are active below that energy. 

\subsubsection{Lattice Methods}
To deal with the non-perturbative sector of the strong interaction the most widespread approach is to use numerical simulations, in particular lattice methods, so it is common to refer to it as Lattice QCD. The main idea, which will be expressed more in detail in Chapter \ref{chap:lattice_intro}, is to discretize space-time and evaluate the field only at fixed sites on a hyper-cubic lattice, physical quantities are then computed stochastically on ensembles of such gauge fields. This approach however is very expensive from a computational point of view and so far no calculation at physical quark masses with a sufficiently small lattice spacing, in principle closer to the continuum theory, have been performed.

\subsubsection{Effective Field Theories}
An interesting problem is to link QCD directly with nuclear forces, that are long range remnants of the strong interaction at a hadron level. The problem is of high interest because there is yet no fundamental theory of nuclear interactions from first principles. The most common approach is to define an Effective Field Theory, starting from a low energy approximation of chromodynamics, that preserves most of the symmetries of the underlying theory. Some common approaches start from considering nucleons as a fundamental $SU(2)$ group in isospin and recognizing pions and kaons as the goldstone bosons of the nuclear forces. 
\NOTE{probably need more...}

\section{Experimental Tests of QCD} 
\NOTE{SOME NICE PLOTS...}