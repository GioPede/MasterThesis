
Lattice QCD is one of the main Lattice Field Theories. It deals with the strong force at low energies, handling non-perturbative calculations in a numerical way. The idea of discretizing space-time in a lattice and perform calculations of field theories was proposed by Wilson in 1974 \cite{wilson_confinement_1974} as an alternative method to explain confinement. It has proven to be the most systematic approach to non-perturbative theories like QCD. In this chapter we will describe briefly how this discretization procedure is performed and some of the main computational strategies that are involved.

\section{Discretizing Field Theories}

The starting point for Lattice QCD is Feynman's path-integral formalism, but expressed in Euclidean space-time, through a Wick rotation. An observable of some field $\phi$ is then given by:
\beq
	\braket{O[\phi]} = \frac{1}{Z[\phi]} \int \D[\phi] ~O[\phi] e^{-S[\phi]}
\eeq
and the euclidean correlator of two quantities $O_1[\phi]$ and $O_2[\phi]$ between times $t_1$ and $t_2$ is:
\beq
    \braket{O_1[\phi](t_1)O_2[\phi](t_2)} = \frac{1}{Z[\phi]} \int \D[\phi] ~O_1[\phi(t_1)]O_2[\phi(t_2)] e^{-S[\phi]}
    \label{lattice:obs_base}
\eeq
where the partition function $Z[\phi]$ is defined as:
\beq
	Z = \int \D[\phi] ~e^{-S[\phi]} 
\eeq
and $S[\phi]$ is the euclidean action of the field:
\beq
S[\phi] = \int d^4x \Lagr [\phi] 
\eeq
Evaluating path-integrals is not possible in general with analytical tools so, in order to allow numerical computations, the Euclidean space-time is discretized on a hyper-cubic lattice $L = (L_x, L_y, L_z, L_t)$. The choice of the lattice spacing, usually denoted $a$, is arbitrary, but most often it is chosen to be equal for all dimensions. If we then define a lattice site $n = (n_x, n_y, n_z, n_t)$ where all the $n$s represent the coordinates of a point in the lattice $\Lambda$, our fields are constrained to have values on the points $an$ instead of on a continuum space-time $x^\mu$.
\beq
    \phi(x) \xrightarrow{discretization} \phi(an)
\eeq
\subsection{The Harmonic Oscillator Example}
\NOTE{Should I include this?}

\section{Discretization of QCD on the Lattice}
In the case of QCD there are two types of field at play, the gluon gauge field $A$ and the $N_f$ fermionic quark fields $\psi$. In order to perform numerical calculations on the lattice, it is necessary to discretize the Euclidean action for both fields. \\LagrIn this section the discretization is performed following a structure that resembles \cref{intro:symmetry}, that is to start from the fermion fields and imposing gauge invariance.

\subsection{Na{\"i}ve Discretization of Fermions}
In this section the fermion discretization procedure found in \cite{gattringer_quantum_2010} is followed, but also some elements from \cite{gupta_introduction_1998}, quoting the main intermediate steps that lead to the formulation of Lattice QCD. The starting point is the fermionic euclidean action in the continuum:
\beq
    S_F [\psi,\bar\psi] = \int d^4x \bpsi(x)(\gamma_\mu\partial^\mu + m)\psi(x)
\eeq 
now we discretize the euclidean space-time on a lattice $\Lambda$ of spacing $a$, each point will be denoted with $n$. The partial derivative can be turned into the central finite difference between neighboring points along the direction of the derivative:
\beq
    \partial_\mu\psi(x) \rightarrow \frac{\psi(n+\hat\mu) - \psi(n-\hat\mu)}{2a}
\eeq 
The discretized fermion action is then the sum of the lagrangian density over all lattice sites multiplied by the unit volume $a^4$:
\beq
    S_F [\psi,\bar\psi] = a^4 \sum_{n\in\Lambda} \bpsi(n) \left[ \sum_{\mu=1}^4\gamma_\mu  \frac{\psi(n+\hat\mu) - \psi(n-\hat\mu)}{2a} + m \psi(n) \right]
\eeq 
As we did in section \ref{intro:symmetry} we try to apply a local gauge transformation $\Omega(n)$ to the field. It is simple to show that some terms of the derivative in the action are not gauge invariant:
\beq
    \bpsi(n)\psi(n\pm\hat\mu) \rightarrow \bpsi(n)\Omega^\dagger(n)\Omega(n\pm\hat\mu)\psi(n\pm\hat\mu)
    \label{eq:fermion_wrong}
\eeq
The problem is in the way the derivative was discretized and the solution is the introduction of an additional field as in \ref{intro:symmetry} that has the correct transformation laws. This field must transform under a local gauge transformation in a way that connects the values of the fermion field at two different lattice sites Because of this property we denote the auxiliary field as ``link variables'', later the matching with the gauge filed will be introduced, and by convention it is written as $U(n, n + \hat\mu) = U_\mu(n)$. Furthermore, it must depend on the orientation along the $\mu$ direction, from \cref{eq:fermion_wrong}:
\beq
U(n, n-\hat\mu) \equiv  U_{-\mu}(n) = U_\mu(n-\hat\mu)^\dagger 
\eeq
In particular, we want it to transform as:
\begin{align}
    \bpsi(n)U_\mu(n)\psi(n+\hat\mu) &\rightarrow \bpsi(n)\Omega^\dagger(n)U'_\mu(n)\Omega(n+\hat\mu)\psi(n\pm\hat\mu)\\\nonumber
    \bpsi(n)U_{-\mu}(n)\psi(n-\hat\mu) &\rightarrow \bpsi(n)\Omega^\dagger(n)U'_{-\mu}(n)\Omega(n+\hat\mu)\psi(n\pm\hat\mu)
\end{align}
from which we infer that:
\begin{align}
    U_\mu(n)    &\rightarrow U'_\mu(n) = \Omega(n)U_\mu(n)\Omega^\dagger(n+\hat\mu)\label{link_transformation}\\\nonumber
    U_{-\mu}(n) &\rightarrow U'_{-\mu}(n) = \Omega(n)U_{-\mu}(n)\Omega^\dagger(n-\hat\mu)
\end{align}
the field $U_\mu(n)$ is an $SU(3)$ valued field, that can be intuitively thought of as a set of oriented links between the sites of the lattice. A visual representation is given in \cref{fig:links}, 

\begin{figure}[!htb]
\centering
\begin{tikzpicture}[node distance=1cm] 
    % arrows
    \draw[->-=.55, line width=1.5pt, >=latex] (2.5,0) to (6.5,0);
    \draw[->-=.55, line width=1.5pt, >=latex] (-2.5,0) to  (-6.5,0);
    
    \draw (6.5, 0.3) -- (6.5, -0.3);
    \draw (2.5, 0.3) -- (2.5, -0.3);
    \draw (-2.5, 0.3) -- (-2.5, -0.3);
    \draw (-6.5, 0.3) -- (-6.5, -0.3);

    \draw (6.5, 0) -- (6.8, 0);
    \draw (2.5, 0) -- (2.2, 0);
    \draw (-2.5, 0) -- (-2.2, 0);
    \draw (-6.5, 0) -- (-6.8, 0);

    \filldraw (6.5, 0) circle (0.05);
    \filldraw (2.5, 0) circle (0.05);
    \filldraw (-2.5, 0) circle (0.05);
    \filldraw (-6.5, 0) circle (0.05); 

    \node [above right] (A) at (6.5, 0) {$n+\hat\mu$};
    \node [above left] (B) at (2.5, 0) {$n$};
    \node [above right] (D) at (-2.5, 0) {$n$};
    \node [above left] (C) at (-6.5, 0) {$n-\hat\mu$};

    \node (E) at (-4.5, -0.7) {$U_{-\mu}(n) \equiv U_\mu^\dagger(n-\hat\mu)$};
    \node (F) at (4.5, -0.7) {$U_{\mu}(n)$};
\end{tikzpicture}
\capt{Schematic representation of the link variables $U_{\mu}(n)$ and $U_{-\mu}(n)$ on the lattice.}
\label{fig:links}
\end{figure}
 
With this results we can write a gauge invariant lattice fermion action as:
\beq
    S_F [\psi,\bar\psi] = a^4 \sum_{n\in\Lambda} \bpsi(n) \left[ \sum_{\mu=1}^4\gamma_\mu  \frac{U_\mu(n)\psi(n+\hat\mu) - U_{-\mu}(n)\psi(n-\hat\mu)}{2a} + m \psi(n) \right]
    \label{intro:lat_action}
\eeq

\subsection{The Gauge Transporter and the Wilson Loop}
For a more formal definition of the link variables we can look at the continuum gauge transporter. This is the path-ordered product, denoted with $\mathcal{P}$ of a gauge field $A_\mu(x)$ along some curve $\mathcal{C}$ between two points $x$ and $y$ in space-time:
\beq
    G(x,y) = \mathcal{P} \exp\left( ig_0\int_\mathcal{C} A_\mu(x')dx'^\mu  \right)
\eeq
An important thing to note now is that gauge transporter belongs to the gauge group, $SU(3)$ for QCD, and not to the algebra, $\mathfrak{su}(3)$. 
Up to order $\mathcal{O}(a)$ the integral on the straight line connecting two points on the lattice can be approximated with $aA_\mu(n)$ for small lattice spacings. For convenience the gauge filed is scaled by a factor $1/g_0$, giving: 
\beq
\tilde A_\mu(x) = \frac{1}{g_0} A_\mu(x)
\label{eq:fieldscale}
\eeq
but for notation's sake we'll drop the tilde from now on. One is then left with:
\begin{align}
    G(n,n+\hat\mu) &\equiv U_\mu(n) ~~= \exp(iaA_\mu(n)) = \mathds{1} + iaA_\mu(n) + \mathcal{O}(a^2) \label{eq:transporter}\\\nonumber
    G(n,n-\hat\mu) &\equiv U_{-\mu}(n) = \exp(-iaA_\mu(n)) = \mathds{1} - iaA_\mu(n-\hat\mu) + \mathcal{O}(a^2)
\end{align}
where in the second equation, for the negative direction, the approximation is taken on the arrival point instead on the initial one. With this definition is easy to see the relation we just stated earlier on the direction change:
\beq
    U(n, n-\hat\mu) = U_{-\mu}(n) = U^\dagger_\mu(n)
\eeq
In the continuum case it can be shown, in \cite{peskin} for example, that the trace of a gauge transporter that has the same start and final point, a Wilson Loop, is gauge invariant. On the lattice the minimal Wilson loop is just a square on a fixed plane $\mu\nu$, and we call this a Plaquette:
\begin{align}
    \label{plaquette}
P_{\mu\nu}(n) &= U_\mu(n) U_\nu(n+\hat\mu) U_{-\mu}(n+\hat\mu+\hat\nu) U_-{\nu}(n+\hat\nu)  \\\nonumber
              &= U_\mu(n) U_\nu(n+\hat\mu) U^\dagger_\mu(n+\hat\nu) U^\dagger_\nu(n)
\end{align}
a pictorial representation is given in \cref{fig:plaq}.
\begin{figure}[!htb]
    \centering
    \begin{tikzpicture}[node distance=1cm, remember picture] 

        \pgfmathsetmacro\xcoord{2.5};
        \pgfmathsetmacro\tiplen{0.4};
        % arrows
        \draw[->-=.55, line width=1.5pt, >=latex] (\xcoord,\xcoord) to (-\xcoord,\xcoord);
        \draw[->-=.55, line width=1.5pt, >=latex] (-\xcoord,\xcoord) to  (-\xcoord,-\xcoord);
        \draw[->-=.55, line width=1.5pt, >=latex] (-\xcoord,-\xcoord) to (\xcoord,-\xcoord);
        \draw[->-=.55, line width=1.5pt, >=latex] (\xcoord,-\xcoord) to  (\xcoord,\xcoord);

        \node (A) at (0, 3.1) {};%{$U_{-\mu}(n+\hat\mu+\hat\nu) = U^\dagger_{\mu}(n+\hat\nu)$};
        \node (B) at (-2.7, 0) {};%{$U_{-\nu}(n+\hat\mu) = U^\dagger_{\nu}(n+\hat\nu)$};
        \node (D) at (0, -3.1) {};%{$U_{\mu}(n)$};
        \node (C) at (2.7, 0) {};%{$U_{\nu}(n+\hat\mu) $};

        \draw (\xcoord, \xcoord) -- (\xcoord, \xcoord+\tiplen);
        \draw (\xcoord, \xcoord) -- (\xcoord+\tiplen, \xcoord);
        \draw (\xcoord, -\xcoord) -- (\xcoord, -\xcoord-\tiplen);
        \draw (\xcoord, -\xcoord) -- (\xcoord+\tiplen, -\xcoord);
        \draw (-\xcoord, -\xcoord) -- (-\xcoord, -\xcoord-\tiplen);
        \draw (-\xcoord, -\xcoord) -- (-\xcoord-\tiplen, -\xcoord);
        \draw (-\xcoord, \xcoord) -- (-\xcoord, \xcoord+\tiplen);
        \draw (-\xcoord, \xcoord) -- (-\xcoord-\tiplen, \xcoord);
        
        \filldraw (\xcoord, \xcoord) circle (0.05);
        \filldraw (\xcoord, -\xcoord) circle (0.05);
        \filldraw (-\xcoord, -\xcoord) circle (0.05);
        \filldraw (-\xcoord, \xcoord) circle (0.05); 
    \end{tikzpicture}

    \begin{tikzpicture}[node distance=1cm, remember picture, overlay] 
        \draw (A) node {$U_{-\mu}(n+\hat\mu+\hat\nu) = U^\dagger_{\mu}(n+\hat\nu)$};
        \draw [left](B)  node {$U_{-\nu}(n+\hat\mu) = U^\dagger_{\nu}(n+\hat\nu)$};
        \draw (D) node {$U_{\mu}(n)$};
        \draw [right](C) node {$U_{\nu}(n+\hat\mu) $};

        \draw[->, >=latex] (-5.7,1.1) to (-5.7,2.6);
        \draw[->, >=latex] (-5.7,1.1) to (-4.1,1.1);

        \node [left] (E) at (-5.7,2.6) {$x_\nu$};
        \node [below] (F) at  (-4.1,1.1) {$x_\mu$};
    \end{tikzpicture}
    \capt{The Plaquette, as defined on the lattice, it terms of the oriented product of the link variables of a minimal square in the plane $\mu\nu$.}
    \label{fig:plaq}
\end{figure}

This allows us to write using \cref{eq:transporter} and applying the Baker-Campbell-Hausdorff formula to deal with the product of matrix exponentials:
\begin{align}
    \nonumber P_{\mu\nu}(n) &=\exp\bigg[iaA_\mu(n) + iaA_\nu(n+\hat\mu) -iaA_\mu(n+\hat\nu) -iaA^\dagger_\nu(n) - \frac{a^2}{2}[A_\mu(n), A_\mu(n+\hat\mu)] \\
    & - \frac{a^2}{2}[A_\nu(n+\hat\nu), A_\nu(n)] + \frac{a^2}{2}[A_\mu(n), A_\nu(n)] + \frac{a^2}{2}[A_\nu(n+\hat\mu), A_\mu(n+\hat\nu)] \label{superlong} \\ 
    \nonumber & + \frac{a^2}{2}[A_\mu(n), A_\mu(n+\hat\nu)] + \frac{a^2}{2}[A_\nu(n+\hat\mu), A_\nu(n)] + \mathcal{O}(a^3)\bigg]
\end{align}
Now the terms that are shifted from the site $n$ can be expanded as:
\beq
    A_\mu(n+\hat\nu) = A_\mu(n) + a\partial_\nu A_\mu(n) +  \mathcal{O}(a^2)
\eeq
and with this substitution most terms cancel and one has:
\begin{align}
    \label{plaq:expanded}
    P_{\mu\nu}(n) &=\exp\left[ ia^2(\partial_\mu A_\nu(n) - \partial_\nu A_\mu(n) + i[A_\mu(n),A_\nu(n)]) + \mathcal{O}(a^3)  \right]\\\nonumber
                  &=\exp\left[ ia^2G_{\mu\nu}(n) + \mathcal{O}(a^3)  \right]
\end{align}
Where we recovered the gauge field strength tensor $G_ {\mu\nu}$. This term can be used to build the euclidean lattice action term for the gluons. In particular, to be able to recover the continuum action, we would like a term of the form:
\beq
    S_G[U] = \frac{a^4}{2g^2}\sum_{n\in\Lambda}\sum_{\mu\nu} \Tr (G_{\mu\nu}(n)^2) \xrightarrow{a\rightarrow 0}  \frac{1}{4g^2} \int d^4x G_{\mu\nu}(x)^2 = S_G[A]
\eeq 
Note that the $1/g_0^2$ term, that might seem strange since it is not in the known continuum Lagrangian is given by the scaling of the field in \cref{eq:fieldscale}. Up to order $\mathcal{O}(a^2) $ this can be obtained, in terms of the link variables, by:
\beq
    S_G[U] = \frac{2}{g^2}\sum_{n\in\Lambda}\sum_{\mu<\nu} \text{Re} \Tr (\mathds{1} - P_{\mu\nu}(n))
    \label{wilsonaction}
\eeq 
Higher order corrections to this action can be computed analytically by considering higher orders in the BCH expansion in \cref{superlong} and in the exponential expansion in \cref{plaq:expanded}.

\subsection{Lattice Fermions} 
The discretization of fermions as discussed in the previous section is incomplete as it leads to unphysical results. It becomes evident if one considers the Fourier transform of the propagator. Let's first rewrite \ref{intro:lat_action} in a more compact way, introducing the lattice Dirac propagator $M_{xy}[U]$. 
\beq
    S_F [\psi,\bar\psi] = a^4\sum_{n\in\Lambda} \bpsi(x)M_{xy}[U]\psi(y)
\eeq
with
\beq
    M_{xy}[U]= \sum_{\mu=1}^4\gamma_\mu  \frac{U_\mu(x)\delta_{x,(y-\hat\mu)} - U^\dagger_{-\mu}(x)\delta_{x,(y+\hat\mu)}}{2a} + m \delta_{x,y} 
\eeq
in momentum space the propagator becomes:
\beq
    \tilde M_{pq}= \delta(p-q)\tilde M(p)~~~~\text{where}~~~~\tilde M(p) = m\mathds{1} + \frac{i}{a}\sum_{\mu=1}^4\gamma_\mu\sin(p_\mu a)
\eeq
In order to calculate the inverse of the propagator in real space we need to invert the one in momentum space and perform an inverse Fourier transform. However, the inverse of the propagator in Fourier space has multiple poles, for example in the case of massless fermions:
\beq
\tilde M(p)^{-1}\bigg\rvert_{m=0} =  \frac{-ia\sum_{\mu}\gamma_\mu\sin(p_\mu a)}{\sum_{\mu}\sin^2(p_\mu a)}
\eeq
the problem vanishes for $a\rightarrow 0$, the continuum case, returning just one fermion type, but on the lattice multiple fermions, one for each pole emerge. This is known as the ``doubling problem'' and the extra fermions are called doublers. The solution, proposed by Wilson \cite{wilson_confinement_1974}, is to modify the propagator adding a term that makes the poles in the inverse of the Fourier transformed propagator vanish. In momentum space:
\beq
M^W(p) = m\mathds{1} + \frac{i}{a}\sum_{\mu=1}^4\gamma_\mu\sin(p_\mu a) + \mathds{1}\frac{1}{a}\sum_{\mu=1}^4 (1-\cos(p_\mu a))
\eeq
and in coordinate space:
\beq
M^W_{xy}[U]= \frac{1}{2a}\sum_{\mu=\pm1}^{\pm4}(\mathds{1} -\gamma_\mu) U_\mu(x)\delta_{x,(y-\hat\mu)} + \left(m +\frac{4}{a}\right)\delta_{x,y} 
\eeq
the shorthand notation $\gamma_{-\mu} = -\gamma_\mu$ has been introduced. \\
The final form of the Wilson Fermion Action is:
\beq
    S_F [\psi,\bar\psi] = a^4\sum_{n\in\Lambda} \bpsi(x)M^W_{xy}[U]\psi(y)
    \label{intro:ferm_action}
\eeq

Now that all the needed information about the action and the fields is set, mainly through equations \ref{wilsonaction} and \ref{intro:ferm_action}, the picture of how to discretize QCD from the continuum Minkovskian space-time to an euclidean space-time lattice is complete.


\section{Path Integrals on the Lattice}
\label{sec:pathintegral}
To express expectation values and correlators on the lattice, path integral formalism is used. The partition function, as we have seen earlier, is the path integral of the fields over the whole space of the action. For the case of QCD the fields are $U$, $\psi$ and $\bpsi$:
\beq
	Z = \int \D\psi\D\bpsi\D U e^{-S[\psi,\bar{\psi},U] }  
\eeq
with the action being the sum of the gluonic and fermionic parts:
\beq
S[\psi,\bar{\psi},U]=S_G[U] + S_F[\psi,\bar{\psi}, U] = S_G[U] + \sum_f \bpsi M[U] \psi
\eeq
The immediate simplification is to integrate out the fermion fields. As in the continuum case one can perform an integration on the Grassmann-valued fields, in general for an integral over some Grassmann numbers $\theta_i$ and their complex conjugates $\theta_i^*$, and a Hermitean matrix $K$:
\begin{align}
    \label{lattice:grassman}
    \int \D\Theta^*\D\Theta e^{-\Theta K \Theta } &= \left( \prod_i\int d\theta_i^*d\theta_i \right)  e^{-\theta_i^* K_{ij} \theta_j } =  \left( \prod_i\int d\theta_i^*d\theta_i \right)  e^{-\sum_i\theta_i^* k_i \theta_i } \\\nonumber
    &= \prod_i b_i = \det B
\end{align} 
This result is very different from what one would get in the real case, $(2\pi)^n/\det B$. It can also be shown that:

\begin{align}
    \int \D\Theta^*\D\Theta \theta_a^*\theta_b e^{-\Theta* K \Theta } &= \left( \prod_i\int d\theta_i^*d\theta_i \right) \theta_a^*\theta_b e^{-\theta_i^* K_{ij} \theta_j } =  \left( \prod_i\int d\theta_i^*d\theta_i \right) \theta_a^*\theta_b e^{-\sum_i\theta_i^* k_i \theta_i } \\\nonumber
    &=  (\det B ) (B^{-1})_{ab}
\end{align}
This last result is crucial for computing fermion correlators for example, for a given ``source'' $\bpsi(x)$ and a ``sink'' $\psi(y)$ the propagator between the two can be computed via path integrals, but it requires inverting the fermion action matrix. \\
With the result of \ref{lattice:grassman} we can simplify greatly the partition function:
\beq
	Z = \int \D\psi\D\bpsi\D U e^{-S[\psi,\bar{\psi},U] }  = \int \D U e^{-S_G[U] } \det M[U] 
\eeq
In a similar fashion as in statistical mechanics, the expectation value of an observable on the lattice can be computed, recalling the general form \cref{lattice:obs_base} as:
\beq
    \langle O \rangle = \frac{1}{Z}  \int \D U ~O[\psi, \bpsi, U] e^{-S_G[U] } \det M[U] 
    \label{lattice:expectation}
\eeq
The above expression cannot be evaluated or simplified analytically any further, so the usual approach is to approximate the path integral numerically. The main idea is to create an ensemble of field configurations using the probability distribution:
\beq
    P[U] = \frac{e^{-S_G[U]}\det M}{Z}
\eeq 
To reproduce the integral in \cref{lattice:expectation}, on such set $\mathcal{U} = \{ U_1, U_2, \dots,U_N \}$ of configurations one computes the observable, the average value is the expectation value:
\beq
\langle O \rangle =  \int \D U~ P[U] O[\psi, \bpsi, U] \approx \frac{1}{N} \sum_{i=1}^N O(\psi, \bpsi, U_i) 
\eeq

The choice of the set $\mathcal{U}$ is the interesting, but difficult, part which is solved by using Monte Carlo methods. The different configurations are chosen in the most widely accepted solution via a Markov chain. In \cref{chap:code_design} a more detailed description of this problem, and a sketch of the implementation that has been developed, will be presented.

\subsection{Pure Gauge Field Theory}
Computing full QCD on the lattice is computationally expensive, mainly due to the integration of fermions via the determinant. From a numerical point of view, the determinant need to be computed at every step of the Markov chain that is used to evaluate the path-integral and this operation affects the time cost of sampling the configuration space dramatically. A first approach is to neglect the determinant completely, considering it constant. This is effectively removing dynamical fermions, freezing them to the lattice sites. This approximation is usually referred to as ``quenched QCD'', or QQCD. \\
The properties of this theory, that is then reduced to a simple Yang-Mills theory with zero fermion flavors, are still interesting to study and have historically played a very important role, being the only accessible simulation until sufficient computing power became available. For example, from \cref{beta:qcd} one can see that the theory presents confinement and asymptotic freedom. 

\subsection{Observables}
On the lattice, given the transformation \ref{link_transformation}, any product of link variables that starts and end at the same site, a closed loop, is gauge invariant. The average values of these objects over the whole lattice can be linked to physical observables, for example the field tensor. In a more general form any observable $L[U]$ of the type:
\beq
    L[U] = \Tr \left[ \prod_{(n,\mu)\in\mathcal{P}} U_\mu(n)\right]
\eeq
where $\mathcal{P}$ is a closed path of links on the lattice, is a gauge invariant object.

\subsubsection{Plaquette}
The simplest observable, which we have already encountered upon defining the Wilson action in \ref{wilsonaction}, is the plaquette. This is the minimal closed loop on the lattice and its value is related to the coupling constant of the action. For each lattice site there are 12 possible plaquettes to be computed, given all the combinations of euclidean indeces. One can consider the average value of the plaquette on the lattice as an observable, related to the value of the action of the field. A proper definition of the observable is:
\beq
    P[U] = \frac{1}{6V}\sum_{n\in\Lambda}\sum_{\mu<\nu}P_{\mu\nu}(n)
\eeq
where $P_{\mu\nu}(n)$ is the one defined in \cref{plaquette} and $V$ is the number of dimensionless lattice volume (the total number of lattice sites). \\
The plaquette expectation value can be used as a tool to determine when the Markov Chain that is used to generate the ensemble, to measure  other observables, has reached a stationary point of the action. In particular, in \cref{sec:thermalization}, the use of the plaquette for checking thermalization of the chain is discussed.

\subsubsection{Energy Density}
The energy of the field is proportional to the square of the field tensor, so the energy density is in particular:
\beq
    E[U] = -\frac{1}{4V}\sum_{n\in\Lambda}\sum_{\mu<\nu}\text{Re}\Tr(G_{\mu\nu}G^{\mu\nu})
    \label{eq:energy}
\eeq
In order to estimate this quantity, one has to compute the field tensor at every lattice site, square it and sum over the whole space. It is then usually normalized by the lattice volume (the number of sites), to get the density. The simplest definition of the field tensor is $G_{\mu\nu}^{(plaq)} = \mathds{1} - P_{\mu\nu}$ but this is not very accurate. A more symmetric definition can be obtained by the ``clover'' field tensor, that is the sum of all the plaquettes of a same plane $\mu\nu$ that start from a given lattice site take all with the same orientation. 

\begin{figure}[!htb]
    \centering
    \begin{tikzpicture}[node distance=1cm, remember picture] 

        \pgfmathsetmacro\len{2.5};
        \pgfmathsetmacro\start {0.2};

        % ++
        \draw[->-=.55, line width=1.5pt, >=latex] (\start,\start) to (\start+\len,\start);
        \draw[->-=.55, line width=1.5pt, >=latex] (\start+\len,\start) to  (\start+\len,\start+\len);
        \draw[->-=.55, line width=1.5pt, >=latex] (\start+\len,\start+\len) to (\start,\start+\len);
        \draw[->-=.55, line width=1.5pt, >=latex] (\start,\start+\len) to (\start, \start);

        %+-
        \draw[-<-=.55, line width=1.5pt, >=latex] (\start,-\start) to (\start+\len,-\start);
        \draw[-<-=.55, line width=1.5pt, >=latex] (\start+\len,-\start) to  (\start+\len,-\start-\len);
        \draw[-<-=.55, line width=1.5pt, >=latex] (\start+\len,-\start-\len) to (\start,-\start-\len);
        \draw[-<-=.55, line width=1.5pt, >=latex] (\start,-\start-\len) to (\start, -\start);

        %--
        \draw[-<-=.55, line width=1.5pt, >=latex] (-\start,-\start) to (-\start,-\start-\len);
        \draw[-<-=.55, line width=1.5pt, >=latex] (-\start,-\start-\len) to  (-\start-\len,-\start-\len);
        \draw[-<-=.55, line width=1.5pt, >=latex] (-\start-\len,-\start-\len) to (-\start-\len,-\start);
        \draw[-<-=.55, line width=1.5pt, >=latex] (-\start-\len,-\start) to (-\start, -\start);

        %-+
        \draw[->-=.55, line width=1.5pt, >=latex] (-\start,\start) to (-\start,\start+\len);
        \draw[->-=.55, line width=1.5pt, >=latex] (-\start,\start+\len) to  (-\start-\len,\start+\len);
        \draw[->-=.55, line width=1.5pt, >=latex] (-\start-\len,\start+\len) to (-\start-\len,\start);
        \draw[->-=.55, line width=1.5pt, >=latex] (-\start-\len,\start) to (-\start, \start);

        \filldraw (0,0) circle (0.1);
    \end{tikzpicture}


    \begin{tikzpicture}[node distance=1cm, remember picture, overlay] 
        \pgfmathsetmacro\lenarrow{1.2};
        \pgfmathsetmacro\startarrowx{-5.7};
        \pgfmathsetmacro\startarrowy{0.5};
        \draw[->, >=latex] (\startarrowx, \startarrowy) to (\startarrowx, \startarrowy+\lenarrow);
        \draw[->, >=latex] (\startarrowx, \startarrowy) to  (\startarrowx+\lenarrow, \startarrowy);

        \node [left] (E) at  (\startarrowx, \startarrowy+\lenarrow) {$x_\nu$};
        \node [below] (F) at (\startarrowx+\lenarrow, \startarrowy) {$x_\mu$};
    \end{tikzpicture}
    \capt{Schematic representation of the symmetric definition of the clover term $G_{\mu\nu}^{(clover)}$ on the lattice.}
    \label{fig:clover}
\end{figure}

In terms of link variables this is equal to:
\NOTE{CHECK BETTER...}
\begin{align}
    G_{\mu\nu}^{(clover)}(n)=\frac{1}{4}\bigg[  &U_\mu(n)U_\nu(n+\hat\mu)U^\dagger_\mu(n+\hat\nu)U^\dagger_\nu(n)\\\nonumber
    - &U_\nu(n)U^\dagger_\mu(n+\hat\nu)U^\dagger_\nu(n-\hat\mu)U_\mu(n-\hat\mu)\\\nonumber
    + &U^\dagger_\mu(n-\hat\mu)U^\dagger_\nu(n-\hat\mu-\hat\nu)U_\mu(n-\hat\mu-\hat\nu)U_\nu(n-\hat\nu)\\\nonumber
    - &U^\dagger_\nu(n-\hat\nu)U_\mu(n-\hat\nu)U_\nu(n+\hat\mu-\hat\nu)U^\dagger_\mu(n) \bigg]
\end{align}


\subsubsection{Topological Charge}
The gauge fields in QCD exhibit particular topological properties that are believed to have important physical implications \cite{witten_current_1979}\cite{di_giacomo_topology_1997}. The topological charge is an integer quantum number of the field in the continuum, similar to the concept of winding-number. In general fields with different topological charge are separated, but there exist non-zero instanton modes \CIT that allow tunneling between them.
On the lattice certain definitions can be used to reproduce the continuum properties, especially for the so called ``topological susceptibility'', that is the second moment of the distribution of the topological charge, which seems to be independent from the definitions of the base observable \cite{alexandrou_comparison_2017}. 

\NOTE{CHECK...}
In the continuum topological sectors, regions of space with same charge, are separated from each other, on the lattice through discretization effects the behavior is different, with instantons that allow tunneling between sectors \cite{gross_qcd_1981}.\\ 


The topological charge is the integral over all space-time of the topological charge density:
\beq
    Q = \int d^4xq(x)
\eeq
where
\beq 
    q(x)=\frac{1}{64\pi^2}\Tr(G_{\mu\nu}\tilde G^{\mu\nu}) 
\eeq 
with $\tilde G^{\mu\nu} = \frac{1}{2}\epsilon^{\mu\nu\rho\sigma} G^{\rho\sigma}$. This can be estimated on the lattice, in the simplest way, by using the same definition of the field strength tensor we used before for the energy density, the clover:
\beq
    Q[U]=\frac{1}{64\pi^2} \sum_{n\in\Lambda}\sum_{\mu<\nu}\epsilon^{\mu\nu\rho\sigma}\Tr[G_{\mu\nu}^{(clover)}(n)G_{\rho\sigma}^{(clover)}(n)]
\eeq
As the lattice spacing is reduced, approaching the continuum, the topological sectors get more and more separated, preventing tunneling between them. This makes the Markov chain used to generate the ensemble less efficient in terms of growing autocorrelation times, as will be shown in \cref{sec:testautocorr}.\\
A derived quantity of great interest is the topological susceptibility $\chi$, that is proportional to the expectation value of the square of the topological charge:
\beq
    \chi = \frac{(\hbar c)^4}{a^4|\Lambda|}\langle Q^2 \rangle
\eeq
This observable is particularly important for the properties of instantons on the lattice and its value can be related via the Witten-Veneziano formula to the mass of the $\eta'$. \CIT

\section{Modern Lattice QCD Calculations}
\NOTE{motlo vaga...\\}
The state of the art calculations in Lattice QCD involve improved fermion and gluon actions. For the gluon sector the improvements are based on finding linear combinations of gauge invariant loops, such as $2\times 1$, $2\times 2$ and $3\times 1$ rectangles to systematically remove the higher order in the right-hand side of \cref{plaq:expanded}. For fermions the improvements consist on actions that preserve certain symmetries more than others, like chiral fermion actions or staggered fermions.\\
On the algorithm side, the currently most used method to generate gauge field configuration is the Hybrid Monte Carlo method, which combines stochastic sampling with hamiltonian dynamic updates in the  gauge field space. Other interesting algorithmic problems regard the calculation of fermion determinants, which typically involve very large sparse matrix diagonalization.\\
A fundamental concept in Lattice calculations is to recover the continuum limit in a controlled manner at the end of the analysis. The error sources that come into play when discretizing space-time on a fixed lattice are:
\begin{itemize}
    \item Finite Volume Effects: caused by the non infinite domain of the simulation. Usually periodic boundary conditions are implemented to simulate infinite space, but one should always check if the total lattice volume is large enough, especially when dealing with large systems such as multiple baryons.
    \item State Isolation and Signal to Noise: when computing correlators between two points on the lattice it is possible to extract energy  states of hadrons by looking at the exponential decay of the correlator in euclidean time. However this problem is largely affect by random noise making it hard to extract even the ground state most of the times; excited states are rarely considered.
    \item Chiral Limit: by which is intended the limit for which the masses of the quarks, and consequently of the computed hadrons, on the lattice approach the physical masses. One might question why this is even the case and that calculations should be performed at the physical masses only. However, the fermion propagator matrix becomes more and more sparse the lower he mass of the quark is, making the numerical algorithms that should diagonalize it slower to converge. The usual approach is to perform calculations on a set of pion masses (this is the usual reference for this problem) and afterwards the limit for $m_\pi^{(lat)} \rightarrow m_\pi^{(phys)}$ is taken to reproduce the physical observables.
    \item Continuum Limit: perhaps the most obvious thing to do to improve Lattice QCD calculations is to take finer and finer lattice spacings, to approach the continuum case. It should be noted however that this procedure has two downsides. First the lattice spacing in physical units is fixed by the quark masses, and as we previously mentioned, this affects the feasibility of the numerical simulation itself. Secondly, as the lattice approaches the continuum case the autocorrelation of observables in a given ensemble increases rapidly. This behavior, known as ``critical slow-down'' depends on the observable itself, some are more affected then others  The integrated autocorrelation time for the topological charge for example is believed to have either  power-law with a large exponent or even an exponential relation to the lattice spacing. 
\end{itemize}
As one can notice, Lattice QCD is still an open field of research in many aspects: the algorithmic/numerical, the extraction and improvement of the measurement of physical quantities and the theoretical model itself. 

